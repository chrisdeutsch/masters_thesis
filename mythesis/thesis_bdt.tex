\chapter{BDT}
\label{sec:bdt}

\begin{itemize}
\item Description \& Plots of the input variables
  \begin{itemize}
  \item How does etOverPtLeadTrk differ from EMPOverTrkSysP for 1-prong taus?
  \end{itemize}

\item Preselection
  \begin{itemize}
  \item $p_\mathrm{T} > \SI{20}{\GeV}$ (reconstructed \& truth)
  \item $| \eta | < 2.5$ (reconstructed \& truth)
  \item $| \eta | \notin \left[ 1.37, 1.52 \right]$ (reconstructed \& truth)
  \item truth-matched taus
  \end{itemize}

\item Train baseline for comparison (with default outlier removal)
  \begin{itemize}
  \item NTrees 100 (explain)
  \item MinNodeSize 0.1 (explain)
  \item BoostType AdaBoost (explain)
  \item SeparationType GiniIndex (explain)
  \item PruneMethod NoPruning
  \item UseYesNoLeaf False (explain)
  \item AdaBoostBeta 0.2 (explain)
  \item DoBoostMonitor True
  \item nCuts 200 (explain)
  \item MaxDepth 8 (explain)
  \end{itemize}

\item Switch from AdaBoost to GradBoost (Explain why we don't do a full
  optimization of the AdaBoost vs GradBoost setup)

\item Variable transformations (log, clamp, uniformization)
  \begin{align}
    &\text{SumPtTrkFrac:} &\log(x + 10^{-4}) \\
    &\text{absipSigLeadTrk:} &\min(x, 30) \\
    &\text{mEflowApprox:} &\log(x) \\
    &\text{centFrac:} &\min(x, 1) \\
    &\text{innerTrkAvgDist:} &- \\
    &\text{ptRatioEflowApprox:} &\min(x, 4) \\
    &\text{EMPOverTrkSysP:} &\log(\max(10^{-3}, x)) \\
    &\text{etOverPtLeadTrk:} &\log(\max(0.1, x)) \\
    &\text{ChPiEMEOverCaloEME:} &\max(-4, \min(x, 5)) \\
    &\text{trFlightPathSig:} &\log(\max(0.01, x)) \\
    &\text{massTrkSys:} &\log(x) \\
    &\text{dRmax:} &-
  \end{align}

\item BDT optimization for R21 (Total 480 BDTs) -- If enough time try 50\%
  bagging
  \begin{itemize}
  \item[NTrees] 25, 50, 100, 200, 400, 800
  \item[Depth] 4, 6, 8, 12, 18
  \item[Shrinkage] 0.05, 0.1, 0.2, 0.4
  \item[MinNodeSize] (0.001), 0.01, 0.1, 1.0
  \end{itemize}


\item Variable importance (Google: Variable Importance Measures)
  \begin{itemize}
  \item Drop-one test
  \item Partial dependence (can we use a two way partial dependence plot to
    show interaction of pt and another variable?)
    \url{http://scikit-learn.org/stable/modules/ensemble.html#interpretation}
  \end{itemize}

\item $p_\mathrm{T}$-dependency 2D histograms of (centFrac / innerTrkAvgDist --
  worse \& SumPtTrkFrac -- should get better) (Plot 'variable separation' vs.
  pt)

\item Performance on gluon / quark initiated jets

\item Cross check with data (jets only)?

\end{itemize}

%%% Local Variables:
%%% mode: latex
%%% TeX-master: "mythesis"
%%% End:
