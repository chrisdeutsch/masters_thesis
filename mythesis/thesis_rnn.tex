\chapter{Identification of Hadronically Decaying $\tau$-Leptons (NN)}
\label{sec:rnn}

\begin{itemize}
\item Neural Network-based Studies
  \begin{itemize}
  \item MLP Tau-ID (only densely connected layers)
    \begin{itemize}
    \item Comparison with BDT-based ID
    \item Scaling with more data
    \end{itemize}
  \item Track-RNN
    \begin{itemize}
    \item Motivation (i.e. \texttt{SumPtTrkFrac} \& MVA-tracking)
    \item Architecture (mention rough optimisation by hand while monitoring
      validation loss)
    \item Preprocessing
    \item Input variables \& correlation with true (or predicted?) class labels.
      Partial dependence plots? Variable importance?
    \item Validation loss vs. number of tracks
    \item Standalone performance vs. BDT-based ID
    \end{itemize}
  \item Cluster-RNN
    \begin{itemize}
    \item Input variables \& correlation with true class labels. Partial
      dependence plots?
    \item Validation loss vs. number of clusters
    \item Standalone performance
    \item Potential method for rate-reduction at HLT
    \end{itemize}
  \item Combined-RNN
    \begin{itemize}
    \item Architecture
    \item Performance w.r.t. BDT-ID
    \end{itemize}
  \end{itemize}
\end{itemize}




















\begin{itemize}
\item Track-RNN
  \begin{itemize}
  \item Replace $\Delta R_\mathrm{JS}$ with $\Delta \eta$ and $\Delta \varphi$
    (try extrapolated \& non-extrapolated)
  \item Replace pt asymmetry with $pt_\mathrm{JS}$
  \item Do a validation loss vs.\ number of tracks scan
  \item Validation loss vs.\ sample size
  \end{itemize}

\item Cluster-RNN
  \begin{itemize}
  \item
  \end{itemize}

\item Combined-RNN
  \begin{itemize}
  \item
  \end{itemize}
\end{itemize}


%%% Local Variables:
%%% mode: latex
%%% TeX-master: "mythesis"
%%% End:
