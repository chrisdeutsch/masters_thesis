% ~ 12 pages
\chapter{Identification of Hadronically Decaying $\tau$-Leptons (NN)}
\label{sec:rnn}

\section{Identification using Feedforward Neural Networks}
\label{sec:ffnn_id}

\begin{itemize}
\item MLP Tau-ID (only densely connected layers)
  \begin{itemize}
  \item Comparison with BDT-based ID
  \item Scaling with more data
  \end{itemize}
\end{itemize}

\section{Identification using Recurrent Neural Networks}
\label{sec:rnn_id}

\subsection{General Description}
\label{sec:rnn_descr}

State that the RNN is mainly optimised for 1-prong operation and give reasons
for that (i.e.\ higher branching ratio)

\subsection{Track-RNN}
\label{sec:rnn_tracks}

\todo{Choice of $z_0 \sin\theta$ instead of $z_0$ is due to expected larger
  error in forward direction}

\todo{Lower threshold for $p_\mathrm{T}$ is \SI{400}{\mega\electronvolt} already
  in track reconstruction}

\todo{LSTM cell size scan? 8, 12, 16, 24, 32, 48, 64}

\begin{figure}[ht]
  \begin{subfigure}{0.5\textwidth}
    \centering
    \includegraphics{./figures/rnn/ntrk_1p.pdf}
    \subcaption{NTracks for 1-prong taus. \todo{Make x-axis label consistent
        with plot on the right. This should be using pt weights.}}
  \end{subfigure}%
  \begin{subfigure}{0.5\textwidth}
    \centering
    \includegraphics{./figures/rnn/nscan/track_1p.pdf}
    \subcaption{Val.\ loss vs.\ nTracks. Also put 3P in this?}
  \end{subfigure}
  \caption{Tracks associated with a tau}
  \label{fig:rnn_ntracks}
\end{figure}

\begin{itemize}
\item Motivation (i.e. \texttt{SumPtTrkFrac} \& MVA-tracking)
\item Architecture (mention rough optimisation by hand while monitoring
  validation loss)
\item Preprocessing
\item Input variables \& correlation with true (or predicted?) class labels.
  Partial dependence plots? Variable importance?
\item Validation loss vs. number of tracks
\item Standalone performance vs. BDT-based ID
\end{itemize}

\begin{itemize}
\item Replace $\Delta R_\mathrm{JS}$ with $\Delta \eta$ and $\Delta \varphi$
  (try extrapolated \& non-extrapolated)
\item Replace pt asymmetry with $pt_\mathrm{JS}$
\item Do a validation loss vs.\ number of tracks scan
\item Validation loss vs.\ sample size
\end{itemize}

\subsection{Cluster-RNN}
\label{sec:rnn_clusters}

\begin{figure}[ht]
  \begin{subfigure}{0.5\textwidth}
    \centering
    \includegraphics{./figures/rnn/ncls_1p.pdf}
    \subcaption{NCluster for 1-prong taus}
  \end{subfigure}%
  \begin{subfigure}{0.5\textwidth}
    \centering
    \todo{INSERT FIGURE HERE!!!}
    \subcaption{Val.\ loss vs.\ nCluster}
  \end{subfigure}
  \caption{Clusters associated with a tau}
  \label{fig:rnn_nclusters}
\end{figure}

\begin{itemize}
\item Input variables \& correlation with true class labels. Partial
  dependence plots?
\item Validation loss vs. number of clusters
\item Standalone performance
\item Potential method for rate-reduction at HLT
\end{itemize}


\subsection{Combined-RNN}
\label{sec:rnn_combined}

\begin{itemize}
\item Architecture
\item Performance w.r.t. BDT-ID
\end{itemize}


%%% Local Variables:
%%% mode: latex
%%% TeX-master: "mythesis"
%%% End:
