\chapter{Introduction}
\label{sec:intro}

Outline of the thesis:

\section{Introduction}

\begin{itemize}
\item LHC \& Experiments
  \begin{itemize}
  \item Discoveries / Measurements
  \item Past: Run-I, Present: Run-II, Future: HL-LHC \& Challenges
  \end{itemize}

\item $\tau$-Leptons
  \begin{itemize}
  \item Importance (Fermionic coupling of Higgs, Higgs CP, Exotics $Z^\prime$,
    $W^\prime$, Heavy Higgs, SUSY)
  \item $\tau$-decay (hadronic branching fraction, decay modes)
  \item Jets faking taus (necessity of identification algorithms)
  \item Classification of $\tau$ decay modes (motivation)
  \end{itemize}

\item Overview of the thesis structure (one bullet point per chapter).
\end{itemize}

\section{Theoretical Background}

\begin{itemize}
\item The Standard Model
  \begin{itemize}
  \item Features \& Successes
  \item Challenges (neutrino masses, dark matter, matter-antimatter asymmetry,
    gravitation, number of parameters, hierarchy problem, \ldots)
  \item Beyond the Standard Model (SUSY -- preferred coupling to down-type
    fermions for large $\tan\beta$ \textrightarrow $\tau$-leptons)
  \end{itemize}

\item Weak Interaction
\begin{itemize}
\item ?
\end{itemize}

\item Strong Interaction
\begin{itemize}
\item ?
\end{itemize}

\item $\tau$-Leptons
\begin{itemize}
\item Discovery
\item Properties (mass \textrightarrow lep \& had, mean life time
  \textrightarrow no direct detection)
\item $\tau$ Physics
  \begin{itemize}
  \item $\mathrm{Z} \rightarrow \tau \tau$ (background for H$\tau \tau$ and
    useful for performance measurements using tag-and-probe -- semileptonic
    decays)
  \item $\mathrm{H} \rightarrow \tau \tau$ (one of two channels to measure
    the fermionic coupling -- $b \bar{b}$ plagued by multijet background,
    HiggsCPs)
  \item MSSM Higgs (potentially high branching fraction to $\tau$-leptons)
  \item $\mathrm{Z}^\prime$ could preferentially decay into third-generation
    fermions (lepton universality not required).
  \item $\mathrm{W}^\prime$ models with preferential coupling to third-gen.
  \end{itemize}

\end{itemize}


\end{itemize}

\section{ATLAS Experiment and the LHC}

\begin{itemize}
\item LHC

\item ATLAS
  \begin{itemize}
  \item Overview
  \item Inner detector
  \item Calorimeter
  \end{itemize}
\end{itemize}

\section{Hadronically Decaying $\tau$-Leptons}

\begin{itemize}
\item
\end{itemize}

\section{Machine Learning (?)}

\begin{itemize}
\item Own chapter describing the machine learning methods used or include into
  \textit{Theoretical Background}? Alternatively include theory section where
  needed (e.g. have one for the BDT-based studies explaining BDTs [short] and
  one for the RNN studies [longer]).

\item Boosted Decision Trees

\item Neural Networks
  \begin{itemize}
  \item Basics
  \item Recurrent Neural Networks
  \item Long Short-Term Memory \cite{lstm}
  \end{itemize}

\end{itemize}

\section{Identification of Hadronically Decaying $\tau$-Leptons}

\begin{itemize}
\item
\end{itemize}

\section{Decay Mode Classification for Hadronically Decaying}

\begin{itemize}
\item
\end{itemize}



\section{Conclusion}

\begin{itemize}
\item
\end{itemize}

\section{Appendix}

\begin{itemize}
\item MC Samples (Preprod, MC16A)
\end{itemize}
