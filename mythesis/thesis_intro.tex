% ~ 2 pages
\chapter*{Introduction}
\addcontentsline{toc}{chapter}{Introduction}
\label{sec:intro}

The Large Hadron Collider (LHC) is currently the largest collider experiment for
fundamental physics research. With its first data taking run, called Run 1,
spanning a time period from 2010 to 2013 at a $pp$ center-of-mass energy
of~\num{7}\,--\,\SI{8}{\TeV} it allowed the four major experiments ATLAS, CMS,
ALICE and LHCb to probe the frontier of particle physics in proton--proton and
heavy ion collisions, enabling tests of the Standard Model on a previously
unreachable energy scale. The first run lead to major achievements, such as the
discovery of a new scalar boson consistent with the Standard Model Higgs
boson~\cite{higgs_atlas, higgs_cms}, evidence and discovery of the rare flavour
changing neutral current decay~$B_s^0 \to \mu \mu$~\cite{lhcb_bs_mumu,
  cms_bs_mumu} and the continued study of hot and dense strongly interacting
matter in heavy ion collisions at unmatched energy densities. In 2015 data
taking was resumed in Run 2 at a $pp$ center-of-mass energy of \SI{13}{\TeV} at
an increased instantaneous luminosity further improving the physics reach. The
Run 2 data taking period ends in 2018 with the Long~Shutdown~2 in preparation of
Run 3 and the High Luminosity LHC.

The tau lepton is an important part of the physics programme at the LHC. Its
large mass makes it an effective probe of the fermionic coupling of the Higgs
boson. Moreover, an enhanced tau lepton production is predicted in several
theories of physics beyond the Standard Model (BSM). Therefore, the tau lepton
is essential to constrain the parameter space of BSM models. While leptonic
decays of the tau lepton into electrons or muons offer a clean signature in the
detector, the largest fraction of tau decays produces hadrons forming collimated
jets of particles. Jets originating from quarks and gluons, which are more
abundant than tau leptons due to the large multijet production cross section at
the LHC, can fake hadronic tau decays. Therefore, the ATLAS experiment uses a
dedicated identification algorithm~\cite{atlas:taurec:run1, atlas:taurec:run2}
based on multivariate methods utilising track and shower shape variables to
discriminate hadronically decaying taus from jets. A well-performing
identification algorithm is crucial to reduce backgrounds for numerous analyses
involving hadronic tau decays. A future measurement of a potential pseudoscalar
admixture in the Higgs boson via spin correlations in $h \to \tau\tau$
decays~\cite{desch_higgs_cp, harnik, Berge2014} requires selecting hadronic
decay modes of the tau lepton with high purity. In the ATLAS experiment an
algorithm to reconstruct the substructure of hadronic tau decays is employed,
which allows to classify the decay mode and improve the mass reconstruction of
ditau resonances~\cite{atlas:taurec:decaymodes}. The pure selection of the
hadronic decay channels of the tau needed for this measurement, requires an
accurate decay mode classification algorithm.

This thesis is concerned with the optimisation of the tau identification and
decay mode classification algorithms used in the ATLAS experiment for Run 2 of
the LHC. First the existing identification strategy is improved by revising the
variable selection and optimising the configuration of the multivariate
techniques used. The recent advances in artificial neural networks and the
availability of computing resources, facilitate the creation of highly complex
predictive models in domains outside of machine learning research. As a result,
a novel approach of performing tau identification based on sequence learning
techniques is presented. Recurrent neural networks employing sequences of
physics objects reconstructed with the ATLAS detector are used to discriminate
hadronic tau decays from jets. The techniques developed for tau identification
are subsequently applied to the problem of decay mode classification.

The structure of this thesis is given in the following. The theoretical
background and motivation is presented in Chapter~\ref{sec:theory}. A
description of the ATLAS detector is given in Chapter~\ref{chap:atlas} and the
reconstruction of hadronic tau lepton decays in
Chapter~\ref{sec:reconstruction}. Subsequently, the multivariate techniques used
in this thesis including Boosted Decision Trees and neural networks are
described in Chapter~\ref{sec:ml}. The optimisation of the preexisting tau
identification algorithm based on Boosted Decision Trees is given in
Chapter~\ref{sec:bdt}, while the novel method based on recurrent neural networks
is presented in Chapter~\ref{sec:rnn}. Finally, the method is applied to the
decay mode classification in Chapter~\ref{sec:decaymode}. The thesis is
concluded in Chapter~\ref{sec:conclusion}.

\todo[inline]{Short captions}

\todo[inline]{Cleanup citations}

\todo[inline]{Cleanup appendix}

\todo[inline]{In comparative ROC curves: make BDT dashed and RNN same color}

\todo[inline]{Desch interested in pile-up dependence of RNN}

\todo[inline]{Desch interested in migration matrix after Tau-ID}

\todo[inline]{Maybe discuss potential future extensions. E.g.\ do simultaneous
  decay mode classification and jet background reduction}

\todo[inline]{Variable ranking depends on working point.}

%%% Local Variables:
%%% mode: latex
%%% TeX-master: "mythesis"
%%% End:
