\chapter*{Introduction}
\addcontentsline{toc}{chapter}{Introduction}
\label{sec:intro}

Focus more on whats the content of this thesis. (Should give a good overview of
whats going on if only Introduction \& Conclusion is read by the reader)

\begin{itemize}
\item LHC \& Experiments
  \begin{itemize}
  \item Discoveries / Measurements

  \item Past: Run-I, Present: Run-II, Future: HL-LHC \& Challenges
  \end{itemize}

\item $\tau$-Leptons
  \begin{itemize}
  \item Importance (Fermionic coupling of Higgs, Higgs CP, Exotics $Z^\prime$,
    $W^\prime$, Heavy Higgs / SUSY)
  \item $\tau$-decay (hadronic branching fraction, decay modes)
  \item Jets faking taus (necessity of identification algorithms)
  \item Classification of $\tau$ decay modes (motivation)
  \end{itemize}

\item Overview of the thesis structure (one bullet point per chapter).
\end{itemize}

\section{TODO}
\label{sec:TODO}

\begin{itemize}
\item Select BDT from Grid Search for further studies
\item Switch to 'per variable'-based preprocessing
\item Prepare number of Track \& number of Clusters scan
\item Plot nTracks \& nClusters for tau candidates (split 1P \& 3P)
\item LSTM Cell Size Scan $n_c \in \{8, 16, 24, 32, 48, 64\}$ (3P track RNN
  maybe not optimised?)
\item From the RNNs: Group variables into groups, e.g. Impact parameters,
  kinematic quantities, inner detector hits, energy fractions, moments etc. and
  drop whole groups to gauge importance of the variables
\end{itemize}


%%% Local Variables:
%%% mode: latex
%%% TeX-master: "mythesis"
%%% End:
