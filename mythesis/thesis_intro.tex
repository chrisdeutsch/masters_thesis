% ~ 2 pages
\chapter*{Introduction}
\addcontentsline{toc}{chapter}{Introduction}
\label{sec:intro}

Final states containing hadronic tau-lepton decays play important role in ATLAS
physics programme (due to high branching fraction of hadronic modes). Examples:
SM measurements, Higgs boson searches, BSM (extended higgs sector, SUSY),
exotics (heavy gauge bosons), leptoquarks. Also Fermionic Coupling of Higgs,
Higgs CP. These analyses heavily depend on robust reconstruction of the hadronic
decay and particle identification suppressing backgrounds from QCD jets,
electrons and muons faking hadronic tau decays.

Paragraph about jets faking taus (why are identification algorithms necessary).

This thesis summarises the improvements of the algorithm currently used for
tau-identification at the ATLAS experiment. Moreover a novel approach to
identifying hadronic tau-lepton decays based on Recurrent Neural Networks is
presented. Finally the methods are applied to

This thesis is structured as follows: Theoretical Background, ATLAS Experiment
at the LHC, Reconstruction, Energy Calibration and Identification, Machine
Learning, Identification of Hadronic Tau-Lepton Decays using BDTs,
Identification using (Recurrent) Neural Networks, Application in Decay Mode
Classification.

Focus more on whats the content of this thesis. (Should give a good overview of
whats going on if only Introduction \& Conclusion is read by the reader)

\begin{itemize}
\item LHC \& Experiments
  \begin{itemize}
  \item Discoveries / Measurements

  \item Past: Run-I, Present: Run-II, Future: HL-LHC \& Challenges
  \end{itemize}

\item $\tau$-Leptons
  \begin{itemize}
  \item Importance (Fermionic coupling of Higgs, Higgs CP, Exotics $Z^\prime$,
    $W^\prime$, Heavy Higgs / SUSY)
  \item $\tau$-decay (hadronic branching fraction, decay modes)
  \item Jets faking taus (necessity of identification algorithms)
  \item Classification of $\tau$ decay modes (motivation)
  \end{itemize}

\item Overview of the thesis structure (one bullet point per chapter).
\end{itemize}

\todo[inline]{Chapter vs.\ Section???}

\todo[inline]{Train-Test Split and Validation. State that all given metrics are
  always either on the validation or the testing sample.}

\todo[inline]{Consistent usage of tau pt -- visible reconstructed transverse
  momentum?}

\todo[inline]{Introduce tau-identification}

\todo[inline]{Properly define efficiency and rejection?}

\todo[inline]{Short captions}


The theoretical background is described in Chapter~\ref{sec:theory}

The detector in Chapter~\ref{chap:atlas}

Reconstruction and Energy Calibration of Hadronic Tau Lepton Decays in
Chapter~\ref{sec:reconstruction}

MVA in Chapter~\ref{sec:ml}

BDT based studies in Chapter~\ref{sec:bdt}

RNN based studies in Chapter~\ref{sec:rnn}

Decay mode classification~\ref{sec:decaymode}




%%% Local Variables:
%%% mode: latex
%%% TeX-master: "mythesis"
%%% End:
