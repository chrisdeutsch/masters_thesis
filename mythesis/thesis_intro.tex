\chapter*{Introduction}
\addcontentsline{toc}{chapter}{Introduction}
\label{sec:intro}

Outline of the thesis:

\section{Introduction}

\begin{itemize}
\item LHC \& Experiments
  \begin{itemize}
  \item Discoveries / Measurements

  \item Past: Run-I, Present: Run-II, Future: HL-LHC \& Challenges
  \end{itemize}

\item $\tau$-Leptons
  \begin{itemize}
  \item Importance (Fermionic coupling of Higgs, Higgs CP, Exotics $Z^\prime$,
    $W^\prime$, Heavy Higgs, SUSY)
  \item $\tau$-decay (hadronic branching fraction, decay modes)
  \item Jets faking taus (necessity of identification algorithms)
  \item Classification of $\tau$ decay modes (motivation)
  \end{itemize}

\item Overview of the thesis structure (one bullet point per chapter).
\end{itemize}

\section{Theoretical Background}

\begin{itemize}
\item The Standard Model
  \begin{itemize}
  \item Features \& Successes

  \item Challenges (neutrino masses, dark matter, matter-antimatter asymmetry,
    gravitation, number of parameters, hierarchy problem, \ldots)

  \item Beyond the Standard Model (SUSY -- preferred coupling to down-type
    fermions for large $\tan\beta$ \textrightarrow $\tau$-leptons)
  \end{itemize}

\item Weak Interaction
\begin{itemize}
\item ?
\end{itemize}

\item Strong Interaction
\begin{itemize}
\item ?
\item Confinement \& Hadronization
\item Quark \& Gluon initiated jets
\end{itemize}

\item $\tau$-Leptons
\begin{itemize}
\item Discovery

\item Properties (mass \textrightarrow lep \& had, mean life time
  \textrightarrow no direct detection)

\item Hadronically Decaying $\tau$-Leptons
  \begin{itemize}
  \item Feynman Diagram, Decay Modes (interm. resonances) \& Branching Ratios
  \item Detector signature ($\pi^0$ ($\gamma \gamma$ / $\mathrm{e}^+
    \mathrm{e}^- \gamma$), $\pi^\pm$ ($\mathrm{K}^\pm$), $\nu_\tau$)
  \item Jets faking taus
  \item
  \end{itemize}

\item $\tau$ Physics
  \begin{itemize}
  \item $\mathrm{Z} \rightarrow \tau \tau$ (background for H$\tau \tau$ and
    useful for performance measurements using tag-and-probe -- semileptonic
    decays)

  \item $\mathrm{H} \rightarrow \tau \tau$ (one of two channels to measure
    the fermionic coupling -- $b \bar{b}$ plagued by multijet background,
    Higgs CP)

  \item MSSM Higgs (potentially high branching fraction to $\tau$-leptons)

  \item $\mathrm{Z}^\prime$ could preferentially decay into third-generation
    fermions (lepton universality not required).

  \item $\mathrm{W}^\prime$ models with preferential coupling to third-gen.
  \end{itemize}
\end{itemize}

\end{itemize}

\section{Machine Learning (worthy of own chapter?)}

\begin{itemize}
\item Own chapter describing the machine learning methods used or include into
  \textit{Theoretical Background}? Alternatively include theory section where
  needed (e.g. have one for the BDT-based studies explaining BDTs [short] and
  one for the RNN studies [longer]).

\item Boosted Decision Trees
  \begin{itemize}
  \item Boosting: Adaptive Boosting $\alpha$, Gradient Boosting $\eta$
  \item Node splitting: Gini Index
  \item Hyperparameters: $N_\mathrm{Trees}$, $d_\mathrm{Tree}$,
  \end{itemize}

\item Neural Networks
  \begin{itemize}
  \item Basics -- focussed on forward pass (Densely connected layers, Activation)
  \item Training -- Weight initialization, Loss functions, Minibatch Gradient
    Descent, Cross-Validation (training, validation, test split)
  \item Recurrent Neural Networks (RNN Equations, Vanishing Gradient Problem,
    Physics Motivation)
  \item Long Short-Term Memory \cite{lstm} (LSTM Equations)
  \item Technical setup (Keras, theano)
  \end{itemize}
\end{itemize}

\section{ATLAS Experiment and the LHC}

\begin{itemize}
\item LHC (brief)

\item ATLAS
  \begin{itemize}
  \item Overview (Design goals) \\
    brief: Beam Line, Inner Detector \& Solenoid, Calorimeter, Muon System \&
    Toroid, Trigger

  \item Nomenclature (Coordinate system, Pseudorapidity, $\Delta R$)

  \item Inner detector / Tracker
    \begin{itemize}
    \item Pixel, IBL
    \item SCT
    \item TRT
    \item  Transverse Momentum Resolution, Vertex \& Secondary Vertex
      reconstruction, Impact Parameter Resolution, $\eta$-Coverage
    \end{itemize}

  \item Calorimeter
    \begin{itemize}
    \item Presampler, LAr (EM1 - EM3), Had (Tile, LAr)
    \item $\eta$-Coverage, Thickness $X_0$ / $\lambda$, Energy Resolution
      vs. $E$
    \item Topoclusters \& Cluster moments
    \end{itemize}

  \end{itemize}
\end{itemize}

\section{Reconstruction of Hadronically Decaying $\tau$-Leptons at ATLAS}

\begin{itemize}
\item anti-$k_\mathrm{t}$ Jets $R = 0.4$
\item Track Selection \& TJVA
\item MVA tracking
\item Energy \& Calibration
\item Tau-ID (does not fit in here)
\item Substructure Reconstruction (Decay Mode Classification)
\end{itemize}

\section{Identification of Hadronically Decaying $\tau$-Leptons (BDT part)}

\begin{itemize}
\item Features of hadronically decaying $\tau$-leptons vs. Quark/Gluon
  initiated jets

\item Samples
  \begin{itemize}
  \item Gammatautau (Polarization), Dijet JZ1W - JZ6W
  \item Reweighting
  \item Baseline selection ($\eta$, $p_\mathrm{T}$, truth-matching)
  \end{itemize}

\item BDT-based Studies
  \begin{itemize}
  \item Previous setup (incl. explanation \& plots of input variables)
  \item Hyperparameter optimization
    \begin{itemize}
    \item AdaBoost \textrightarrow GradBoost
    \item Grid Search
    \item XGBoost (?)
    \end{itemize}
  \item Variable correlations, importance (dropped variables) \& dependence with
    $p_\mathrm{T}$ (2D Hist), Variable Transformations (instead of cutting out
    outliers), Partial Dependence
  \item Working points (Gammatautau -- Ztautau)
  \item Performance on simulated data
  \item Impact of Quark / Gluon initiated jets on Tau-ID
  \end{itemize}

\end{itemize}

\section{Identification of Hadronically Decaying $\tau$-Leptons (NN-part)}

\begin{itemize}
\item Neural Network-based Studies
  \begin{itemize}
  \item MLP Tau-ID
  \item Track-RNN
    \begin{itemize}
    \item Architecture
    \item Input variables \& correlation with true class labels
    \item Validation loss vs. number of tracks
    \end{itemize}
  \item Cluster-RNN \& Potential rate-reduction at HLT
    \begin{itemize}
    \item Validation loss vs. number of clusters
    \item Input variables \& correlation with true class labels
    \end{itemize}
  \item Combined-RNN
  \end{itemize}
\end{itemize}

\section{Decay Mode Classification for Hadronically Decaying}

\begin{itemize}
\item Signature of the decay modes
\item Explain PFO
\item RNN
  \begin{itemize}
  \item Only PFOs
  \item PFOs + Extras
  \end{itemize}
\end{itemize}

\section{Conclusion}

\begin{itemize}
\item
\end{itemize}

\section{Appendix}

\begin{itemize}
\item MC Samples (Preprod, MC16A)
\end{itemize}

%%% Local Variables:
%%% mode: latex
%%% TeX-master: "mythesis"
%%% End:
