% ~ 2 pages
\chapter*{Introduction}
\addcontentsline{toc}{chapter}{Introduction}
\label{sec:intro}

The Large Hadron Collider (LHC) is currently the largest collider experiment for
fundamental physics research. With its first data taking run, called Run 1,
spanning a time period from 2010 to 2013 at a center-of-mass energy
of~\num{7}\,--\,\SI{8}{\TeV} it allowed the four major experiments ATLAS, CMS,
ALICE and LHCb to probe the frontier of particle physics in proton--proton (and
heavy ion) collisions, enabling tests of the Standard Model on a previously
unreachable energy scale. The first run lead to major achievements, such as the
discovery of a new scalar boson consistent with the Standard Model Higgs
boson~\cite{higgs_atlas, higgs_cms}, evidence and discovery of the rare flavour
changing neutral current decay~$B_s^0 \to \mu \mu$~\cite{lhcb_bs_mumu,
  cms_bs_mumu} and the continued study of hot and dense strongly interacting
matter in heavy ion collisions at unmatched energy densities. In 2015 data
taking was resumed in Run 2 at a $pp$ center-of-mass energy of \SI{13}{\TeV} at
an increased instantaneous luminosity further improving the physics reach. The
Run 2 data taking period ends in 2018 with the Long~Shutdown~2 in preparation of
Run 3 and the High Luminosity LHC.

The tau lepton is an important part of the physics programme at the LHC. Its
large mass makes it an effective probe of the fermionic coupling of the Higgs
boson. Moreover, an enhanced tau lepton production is predicted in several
theories of physics beyond the Standard Model (BSM). Therefore, the tau lepton
is essential to constrain the parameter space of BSM models. While leptonic
decays of the tau lepton into electrons or muons offer a clean signature in the
detector, the largest fraction of tau decays produces hadrons forming a
collimated jet of particles. Jets originating from quarks and gluons, which are
more abundant than tau leptons due to the large multijet production cross
section at the LHC, can fake hadronic tau decays. Therefore, the ATLAS
experiment uses a dedicated identification algorithm based on multivariate
methods utilising track and shower shape variables to discriminate hadronically
decaying taus from jets. A well-performing identification algorithm is crucial
to reduce backgrounds for numerous analyses involving hadronic tau decays.

This thesis

mention: Run 2, center-of-mass energy, identification, classification

Tau decays before entering the active detector volume -- allows measuring the polarisation
Measurements of tau polarisation in ditau resonances require reconstruction of four momentum of constituents of the tau decay
Requires to classify the decay mode of the tau

- Whats done in this thesis
Update of the BDT based Tau-ID
The developments in the field of neural networks and the increasing computing power allow to easily create highly complex models
Using novel machine learning techniques based on recurrent neural networks to improve identification
Applying this to decay mode classification.


- Small preview for each chapter


Final states containing hadronic tau-lepton decays play important role in ATLAS
physics programme (due to high branching fraction of hadronic modes). Examples:
SM measurements, Higgs boson searches, BSM (extended higgs sector, SUSY),
exotics (heavy gauge bosons), leptoquarks. Also Fermionic Coupling of Higgs,
Higgs CP. These analyses heavily depend on robust reconstruction of the hadronic
decay and particle identification suppressing backgrounds from QCD jets,
electrons and muons faking hadronic tau decays.

The abundance of quark- and gluon-initiated jets due to the large multijet
production cross section at the Large Hadron Collider (LHC).

Paragraph about jets faking taus (why are identification algorithms necessary).

This thesis summarises the improvements of the algorithm currently used for
tau-identification at the ATLAS experiment. Moreover a novel approach to
identifying hadronic tau-lepton decays based on Recurrent Neural Networks is
presented. Finally the methods are applied to

This thesis is structured as follows: Theoretical Background, ATLAS Experiment
at the LHC, Reconstruction, Energy Calibration and Identification, Machine
Learning, Identification of Hadronic Tau-Lepton Decays using BDTs,
Identification using (Recurrent) Neural Networks, Application in Decay Mode
Classification.

Focus more on whats the content of this thesis. (Should give a good overview of
whats going on if only Introduction \& Conclusion is read by the reader)


The theoretical background is described in Chapter~\ref{sec:theory}

The detector in Chapter~\ref{chap:atlas}

Reconstruction and Energy Calibration of Hadronic Tau Lepton Decays in
Chapter~\ref{sec:reconstruction}

MVA in Chapter~\ref{sec:ml}

BDT based studies in Chapter~\ref{sec:bdt}

RNN based studies in Chapter~\ref{sec:rnn}

Decay mode classification~\ref{sec:decaymode}

\todo[inline]{Short captions}


%%% Local Variables:
%%% mode: latex
%%% TeX-master: "mythesis"
%%% End:
