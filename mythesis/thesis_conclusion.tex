% ~ 1-2 pages
\chapter{Summary and Outlook}
\label{sec:conclusion}

The performance of the reconstruction and identification algorithms for hadronic
tau lepton decays are of large importance in the ATLAS experiment. They form the
foundation for many physics analyses involving tau leptons. As a result,
improving the performance of these algorithms can lead to significant benefits
for numerous analyses.

In this thesis the tau identification algorithm currently employed in the ATLAS
experiment is improved. The configuration of the BDT used for identification is
systematically optimised and the variable selection revised. The background
rejection is improved by \num{10} to \SI{30}{\percent} for \tauhadvis \pt below
\SI{200}{\GeV} at the fixed signal efficiency working points. For fake
\tauhadvis candidates with \pt above \SI{200}{\GeV} the rejection is improved by
up to \SI{80}{\percent}.
% No increase in inherent complexity?

A method of performing tau identification based on neural networks is presented.
For this the BDT-based identification is reproduced using a simple neural
network. Subsequently, a novel approach employing sequence learning techniques
to discriminate hadronic tau lepton decays from quark- and gluon-initiated jets
is presented. Sequences of charged-particle tracks associated with a \tauhadvis
candidate are combined in a recurrent neural network to reject fake \tauhadvis
candidates. The model presented in this thesis shows performance characteristics
similar to the optimised BDT-based identification using high-level variables,
while almost exclusively using tracking information. The background rejection of
this model exceeds (misses) the rejection of the optimised 1-prong (3-prong)
identification by \num{10} to \SI{20}{\percent}, indicating that the current
algorithm does not fully exploit the isolation information in the track system.

Similarly, the approach is applied to reconstructed clusters of energy in the
calorimeter system. The rejection of a purely cluster-based identification is
insufficient for offline tau identification. A study is presented showing the
benefits of using TopoClusters in the \emph{Global Trigger System} of the ATLAS
detector for the HL-LHC to reduce the trigger rates of a ditau trigger. The
target rate of \SI{200}{\kilo\hertz} could be achieved with \tauhadvis \pt
thresholds of \SI{25}{\GeV} and ditau efficiencies of approximately
\SI{90}{\percent} on the plateau starting at true \tauhadvis \pt of
\SI{40}{\GeV}.

The tau identification studies are concluded with a model combining track,
cluster and high-level variables. The model shows a significant increase in
rejection of approximately \SI{50}{\percent} for fake \tauhadvis candidates at
low \pt. The improvement in background rejection exceeds a factor of two
starting at \tauhadvis \pt of \SI{60}{\GeV} (\SI{100}{\GeV}) for the 1-prong
(3-prong) identification. Analyses with large backgrounds from quark- or
gluon-initiated jets could directly benefit from the improved background
rejection. In cases where the rejection of the current identification is
sufficient, an identification working point with higher signal efficiency can be
employed. In an exemplary analysis measuring a ditau resonance that requires two
tight \tauhadvis, the signal yield of the $\tauhad\tauhad$ channel could be
increased by approximately \SI{30}{\percent}.

Finally, the sequence learning techniques developed in this thesis are applied
to the problem of decay mode classification of hadronic tau decays. For this
constituents of the decay reconstructed with the \emph{Tau Particle Flow}
algorithm are combined in a recurrent neural network to classify the decay mode.
The network is well-suited to perform multiclass classification showing an
improvement in classification accuracy from \num{73.2} to \SI{78.0}{\percent},
thus reducing the number of misclassified decays by almost \SI{20}{\percent}.
The model is extended by including conversion track, shot and neutral PFO
cluster information further improving the accuracy to \SI{81.0}{\percent} and
reducing the misclassifications by approximately \SI{30}{\percent}. The
improvement in classification accuracy could improve the energy calibrations
used for hadronic tau decays, leading to more accurate mass reconstruction of
tau resonances, thus improving the signal extraction process in analyses. For
future measurements of the \cp nature of the Higgs boson it allows to select
hadronic decay modes more efficiently and with higher purity.

Continued work on the algorithms presented in this thesis is planned. While the
studies on the BDT-based identification are largely implemented in the ATLAS
reconstruction framework, the models based on neural networks still allow for
further optimisations. The network architectures allow for countless variations
that could further improve the performance of the identification. The validation
of the network performance on collision data is still pending. Potential
differences in simulation and data could be resolved by training the tau
identification using background samples collected with dijet triggers, which was
done for the identification in Run 1. The infrastructure necessary to evaluate
the models in the ATLAS reconstruction framework is actively being worked on,
such that the models presented in this thesis could be available to analyses of
the full Run~2 collision data.

%%% Local Variables:
%%% mode: latex
%%% TeX-master: "mythesis"
%%% End:
