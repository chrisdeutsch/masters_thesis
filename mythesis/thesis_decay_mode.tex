% ~ 8-10 pages
\chapter{Decay Mode Classification of Hadronically Decaying $\tau$-Leptons}
\label{sec:decaymode}

\section{Tau Particle Flow}
\label{sec:tau_pflow}

CellBased

Tau Particle Flow as described in \cite{atlas:taurec:decaymodes}.

Tau particle flow tries to reconstruct energy contributions of individual
neutral and charged pions in the calorimeter using particle flow. First charged
pions (PFOs) are reconstructed (including momentum and charge) from 'charged
tracks' (MVA tracking) in the inner detector (in Tau Particle Flow these objects
are considered to be charged pions). The charged pions initiate extensive
hadronic showers in the calorimeter.

To reconstruct neutral pions which mostly deposit energy in single (compact)
showers (highly collimated photons) in (PS,) EM1 and EM2. Neutral pions are
therefore reconstructed by clustering cells in PS, EM1, EM2. In case of
overlapping showers of the neutral and charged constituents an energy
subtraction scheme has to be employed.

Due to the decrease in momentum resolution in the tracker as well as the
additional boost of the decay products of the tau leading to merging of
$\pi^0$-clusters, this method is optimized for operation in the low-momentum
regime $p_\text{T} < \SI{100}{\giga\electronvolt}$.

The individual reconstructed hadrons can be used to improve decay mode
classification and energy resolution of reconstructed hadronic tau decays.


\section{Architecture}
\label{sec:pfo_architecture}


\section{Baseline}
\label{sec:pfo_baseline}

'Global' Information attached to every PFO:
\begin{itemize}
\item $p_\text{T}^\text{jet}$
\item $\phi_\text{jet}$
\item $\eta_\text{jet}$
\end{itemize}

Charged PFOs:
\begin{itemize}
\item $p_\text{T}^\text{PFO}$
\item $\Delta \phi$: The angle between PFO and jet in the transverse plane.
\item $\Delta \eta = \eta_\text{PFO} - \eta_\text{jet}$: Angle between PFO and
  jet in the pseudorapidity plane
\end{itemize}

Neutral PFOs (Charged PFOs + whats listed here):
\begin{itemize}
\item $S_\text{BDT}^{\pi^0}$ / $p_{\pi^0}$: Probability that the particle flow
  object originates from a neutral pion (ALSO IN BASELINE)
\item $N_\text{shot}$: \texttt{cellBased\_NHitsInEM1} (Number of shots) Sum of
  $N_\text{photon}$ over all shots matched to the cluster (ALSO IN BASELINE)
\item $\langle R^2 \rangle$: Transverse cluster moment
\item $\langle (\eta - \eta_\text{Cluster})^2 \rangle$: Second moment of $\eta$
  with respect to the cluster position.
\item $N_\text{EM1}^\text{pos}$: Number of positive cells in EM1
\item $f_\text{core}$: \texttt{ENG\_FRAC\_CORE} Fraction of energy in the three
  hottest cells in each sampling \todo{Check}
\item $f_\text{EM1} = E_\text{EM1} / E$: Energy fraction in EM2
\end{itemize}

Preprocessing:
\begin{itemize}
\item $p_\text{T}$: log-transform and standard scaling
\item $\eta$, $\phi$: Divided by maximum value ($2.5$ or $\pi$)
\item $\Delta \eta$, $\Delta \phi$: standard scaling
\item Cluster moments: Second\_R, secondEtaWRTClusterPosition, NPosECells\_EM1
  ... standard scaling
\item PI0BDT, NHitsinEM1, ENG\_FRAC\_CORE, energyfrac em2 - no scaling
\end{itemize}

Conversion tracks: same as charged PFOs (but with tracks) \\
Shots: same as charged PFOs (but with shots) \\



% ----------------------------- FIRST DRAFT -------------------------- %
\begin{itemize}
\item Signature of the decay modes (Could be moved to \textit{Theoretical
    Background})
\item Tau Particle Flow \& Currently used decay mode reconstruction (PanTau)
\item RNN
  \begin{itemize}
  \item Only PFOs
  \item PFOs + Cluster-Moments + Conversion Tracks + Shots
  \item Class probabilities, Migration matrices (row + col. norm),
    Efficiency/Purity profiles
  \item Improvements for Higgs CP measurement in
    $H \rightarrow \tau\tau \rightarrow \rho \rho 2\nu$
  \end{itemize}
\end{itemize}

% ------------------------------- ??? ------------------------------ %
\begin{itemize}
\item PFO-RNN

\item Performance when adding conversion tracks

\item Performance when adding shot information

\item Performance when adding hadronic PFOs

\item Performance when adding more cluster information

\end{itemize}

%%% Local Variables:
%%% mode: latex
%%% TeX-master: "mythesis"
%%% End:
