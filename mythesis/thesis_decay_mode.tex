% ~ 8-10 pages
\chapter{Decay Mode Classification of Hadronic Tau Lepton Decays}
\label{sec:decaymode}

In this chapter the sequence learning techniques developed for
tau-identification are applied to the problem of decay mode classification of
hadronic tau lepton decays. In the following the decay modes containing one
charged hadron $h^\pm$ with one $h^\pm \pi^0$ or more than one neutral pion
$h^\pm \geq 2\pi^0$ as well as three charged hadrons without neutrals $3h^\pm$
or at least one neutral pion $3h^\pm \geq 1\pi^0$ shall be discriminated. The
hadron $h^\pm$ includes charged pions as well as kaons while decays over
intermediate $\text{K}^0$ are omitted. These five decay modes make up the
majority of hadronic tau decays (approx. \SI{92}{\percent}). Neural networks
naturally extend from binary to multi-class classification problems making them
well-suited for the discrimination of the hadronic decay modes.

At first a quick overview of the \emph{Tau Particle Flow}-algorithm is given,
which is used to reconstruct neutral and charged decay products. Subsequently a
neural network is developed that uses the reconstructed decay products to
classify the decay mode into one of five categories. Finally the network is
extended by including additional information from conversion tracks, cluster
moments and shot information. In the following the notation established in
Ref.~\cite{atlas:taurec:decaymodes} will be adopted.

\todo[inline]{Signature of the decay modes (Could be moved to
  \textit{Theoretical Background})}

\todo[inline]{Readable text in migration matrices when printed
  -- dark blue???}

\section{Tau Particle Flow}
\label{sec:tau_pflow}

The \emph{Tau Particle Flow}-algorithm is a specialised particle flow algorithm
for reconstruction of charged and neutral constituents in hadronic tau decays
with visible transverse momenta of up to \SI{100}{\giga\electronvolt}. It aims
to improve reconstruction of individual particles by optimally combining the
information in several subdetector-systems. The reconstructed objects, called
charged or neutral \emph{particle flow objects}~(PFOs), can be used to classify
the hadronic decay modes and for improving the energy resolution of the
reconstructed hadronic tau decay by employing a decay mode specific calibration.
The following description of the algorithm is based on
Ref.~\cite{atlas:taurec:decaymodes} including recent changes to the reconstruction
algorithms.

Charged PFOs are reconstructed from tracks classified as \emph{charged}
according to the track classification. The charge and transverse momentum of the
reconstructed PFO is determined from the measurement in the tracking system,
which has superior energy resolution for charged pions with
$p_\text{T} < \SI{100}{\giga\electronvolt}$ compared to a calorimeter-based
measurement~\cite{atlas:taurec:decaymodes}. The $\pi^\pm$-mass hypothesis is
used to calculate the energy of the PFO.
% Charged hadrons initiate extensive
% hadronic showers depositing most of their energy in the hadronic calorimeters
% (incl.\ EM3) \todo{Motivation for this sentence?}.

Neutral pions often deposit their energy in a single cluster in the EM
calorimeter (Presampler, EM1/EM2) caused by two collimated photons from the
$\pi^0$ decay. Therefore neutral PFOs are reconstructed by clustering all cells
in the electromagnetic part of the calorimeter within the $\Delta R < 0.4$ cone
about the tau axis \todo{Check instead of core-region?}. If a cluster is in the
proximity of a charged PFO then the energy deposition of the charged hadron in
the EM calorimeter has to be separated from the neutral pion energy. The energy
$E_{h^\pm}^{\text{EM}}$ that needs to be subtracted to remove the contribution
of the charged hadron is estimated by
\begin{align*}
  E_{h^\pm}^{\text{EM}} = E_{h^\pm}^{\text{track}} - E_{h^\pm}^{\text{HAD}} \eqcomma
\end{align*}
where $E_{h^\pm}^{\text{track}}$ is the energy of the charged PFO measured in
the tracking system and $E_{h^\pm}^{\text{HAD}}$ the energy of the charged PFO
deposited in the hadronic part of the calorimeter. $E_{h^\pm}^{\text{HAD}}$ is
calculated by matching clustered energy deposits in the HCAL to the closest
track of a charged PFO. The contribution of the charged hadron in the EM
calorimeter~$E_{h^\pm}^{\text{EM}}$ is subtracted from the closest neutral PFO
cluster if the angular distance between cluster and extrapolated track is
smaller than $\Delta R < 0.04$. \todo{Zero mass hypothesis for neutrals?}
Neutral PFOs can often be reconstructed from an incomplete subtraction of the
charged hadron energy deposition in the EM calorimeter or by pile-up. For decay
mode classification it is necessary to identify the neutral pions in all
reconstructed neutral PFOs of the tau decay. The identification exploits the
difference in shower shape of hadronic showers initiated by charged hadrons and
compact showers from photons of the $\pi^0$ decay using multivariate methods.

The number of identified neutral pions can be used for a preliminary
classification of the decay mode. The following sections will be concerned with
combining reconstructed PFOs in neural networks to achieve better classification
power. Decay mode classification employing the \emph{Tau Particle
  Flow}-algorithm is optimised for operation in the low-momentum regime due to
the decreasing momentum resolution in the tracking system as well as the
additional boost of the tau decay products leading to merging of
$\pi^0$-clusters. Therefore this chapter focuses focus on the classification of
hadronic tau lepton decays with visible transverse
momenta~$p_\text{T} < \SI{100}{\giga\electronvolt}$ \todo{High momentum
  performance is also shown!}.

\section{Classification with Tau Particle Flow and Recurrent Neural Networks}
\todo{Better name for this part}
\label{sec:pfo_general}

The charged and neutral PFOs reconstructed with the \emph{Tau Particle
  Flow}-algorithm contain information about the daughter particles of the tau
decay and can be used for decay mode classification. Properties of charged and
neutral PFOs are combined in recurrent neural networks to perform multi-class
classification. Mainly kinematic information (e.g.\ transverse momentum, angular
deviation from the tau axis) is used to describe each PFO. Moreover the
$\pi^0$-likeness in form of the neutral pion identification BDT score is
included in the classification to be able to discriminate between neutral PFOs
originating from a $\pi^0$ and remnants of the subtraction or pile-up.

\begin{figure}[htb]
  \centering
  \includegraphics{./figures/decay_mode_classification/baseline_architecture.pdf}
  \caption{Architecture. Highlighted in blue are layers that operate on a sequence of inputs.}
  \label{fig:pfo_rnn_baseline_arch}
\end{figure}

The network architecture used for the decay mode classification is shown in
Figure~\ref{fig:pfo_rnn_baseline_arch} and is similar to the networks used for
tau-identification in the previous chapter. At the input layer the network is
split into two branches each accepting a number of charged and neutral PFOs
respectively. The shared dense layer applies an affine transformation on the
input variables of each PFO. Afterwards the input sequences are fed into the
recurrent LSTM layer which subsequently return a single vector of activations.
The activations of both branches are then merged and passed through a network
containing three dense layers. The final dense layer has five units equal to the
number of decay modes to classify. The Softmax activation function is applied to
the final layer, ensuring that the output activations sum to one.

\subsection{Baseline Model}
\label{sec:pfo_baseline}

For training and evaluation of the model a sample containing the process
$\gamma^* \rightarrow \tau \tau \, \text{(hadr.)}$ is used (full sample
identifier in Appendix~\ref{sec:app_mc16a_taus}). The baseline tau selection
given in Section~\ref{sec:bdt_eventsim} is applied and the visible transverse
momentum~$p_\text{T}$ of the hadronic decay is required to be less than
\SI{100}{\giga\electronvolt} at reconstruction and generator-level. Moreover
only tau-leptons with the true decay mode being one of $h^\pm$, $h^\pm \pi^0$,
$h^\pm \geq 2\pi^0$, $3h^\pm$ or $3h^\pm \geq 1\pi^0$ and omitting decay chains
containing intermediate $\text{K}^0$ are used. The mode composition of the
reconstructed tau decays after these selections is summarised in
Table~\ref{tab:mode_reco_eff}.

\begin{table}[htb]
  \centering
  \begin{tabular}{
  l
  S[table-format=2.2(2)]
  S[table-format=2.1, round-mode=places, round-precision=1]
  S[table-format=2.1, round-mode=places, round-precision=1]
  }
  \toprule
  {Mode} & {$\mathcal{B}$ / \si{\percent}} & {$f_\text{reco}$ / \si{\percent}} & { $f_\text{reco+ID}$ / \si{\percent}} \\
  \midrule
  $h^\pm$ & 10.51 +- 0.05 & 17.552988 & 21.365416 \\
  $h^\pm \pi^0$ & 25.93 +- 0.09 & 41.854581 & 44.779216 \\
  $h^\pm \geq 2 \pi^0$ & 10.81 +- 0.09 & 18.663134 & 15.775690 \\
  $3 h^\pm$ & 9.43 +- 0.05 & 14.200414 & 13.315444 \\
  $3 h^\pm \geq 1 \pi^0$ & 5.09 +- 0.05 & 7.728881 & 4.764232 \\
  \bottomrule
\end{tabular}

%%% Local Variables:
%%% mode: latex
%%% TeX-master: "../mythesis"
%%% End:

  \caption{Mode reconstruction efficiencies. $h^\pm$ can be pion or kaon.
    Intermediate decays via neutral kaons are excluded. Branching fraction
    $\mathcal{B}$; Mode fractions of reconstructed taus passing preselection
    $f_\text{reco}$; Mode fraction of taus also passing medium tau id.
    $f_\text{BR}$ fraction assuming fully efficient reconstruction.}
  \label{tab:mode_reco_eff}
\end{table}

The discrimination of the decay modes utilises kinematic quantities of the
reconstructed tau decay as well as the charged and neutral particle flow
objects:
\begin{description}
\item[Kinematic quantities of the tau decay:] The visible transverse
  momentum~$p_\text{T}^\tau$ and the angular
  direction~$(\varphi_\tau, \eta_\tau)$ of the reconstructed tau-axis. The
  momentum is calibrated at LC-scale without applying a tau-specific energy
  calibration.

\item[Kinematic quantities of charged and neutral PFOs:] The reconstructed
  transverse momentum $p_\text{T}^\text{PFO}$ and the signed angular distances
  to the reconstructed tau decay in transverse~$\Delta\varphi$ and longitudinal
  direction~$\Delta\eta \coloneqq \eta_\text{PFO} - \eta_\tau$. The transverse
  angular distance~$\Delta\varphi$ is defined analogously to the longitudinal
  case but also accounting for the periodicity in~$\varphi$. Signed angular
  distances relative to the tau-axis are used to ensure that coordinates are
  comparable between different tau candidates.
\end{description}
The variable set used for neutral PFOs is extended to allow for the
identification of clusters originating from neutral pions. In addition to the
kinematic quantities the following variables are used:
\begin{description}
\item[$\pi^0$ identification score $S_\text{BDT}^{\pi^0}$:] BDT-based
  discriminant combining cluster information in the electromagnetic part of the
  calorimeter to identify neutral PFOs originating from the $\pi^0$
  decay~\cite{atlas:taurec:decaymodes}.

\item[Number of shots $N_\text{shots}$:] Number of shots (cf.\
  Section~\ref{sec:shot_reco}) associated with a neutral PFO cluster. Shots are
  associated with a cluster if it contains the seed cell of the shot and the
  cluster fulfils $E_\text{T} > \SI{500}{\mega\electronvolt}$ and is within
  $\Delta R < 0.4$ to the tau axis. In cases where the cell is shared between
  multiple clusters, it is associated to the cluster to which the seed cell
  contributes with the largest weight.

  The fine segmentation of the strip layer of the EM calorimeter is used to
  count local energy maxima created by individual photons and allow to recover
  the correct number of neutrals in cases where the energy depositions of two
  neutral pions are reconstructed as a single cluster in the calorimeter.
\end{description}
\todo[inline]{Give reasons for this variable selection. \SI{1.5}{\percent}
  improvement for shots}

\todo[inline]{Preprocessing:} The transverse momenta of the reconstructed tau as
well as the PFOs are log-transformed and subsequently standardised (by
subtracting the mean and dividing by the standard deviation of the transformed
variable). The remaining kinematic variables are transformed to fall into the
the $[-1, 1]$ range. For the neutral PFO specific variables
$S_\text{BDT}^{\pi^0}$ and $N_\text{shots}$ no preprocessing is needed.

Charged and neutral PFOs are passed in ascending transverse momentum ordering to
the shared dense layers with 24 units each \todo{Up to 3 charged PFOs (upper
  limit for given selection), Up to 10 neutral PFOs -- not optimised}. The
intermediate representation of the input sequence is then passed into the LSTM
layers each with 24 units and hard sigmoid recurrent activation. The recurrent
layers return a single element containing 24 scalar values which are
subsequently merged and passed through three dense layers with 48, 32 and 5
units respectively. The outputs of the different layers are activated using the
$\tanh$ activation function with the exception of the last dense layer, which
uses \emph{Softmax} activation.

For each reconstructed tau decay the model returns one probability for each of
the five decay modes. The reconstructed decay mode is then given as the mode
with the highest probability according to the model. A different scheme for
determining the decay mode would be to require the largest mode probability to
exceed the second largest by a predefined margin. This would increase the purity
of the reconstructed modes at the expense of reducing the reconstruction
efficiency (should be evaluated on analysis-level).

In contrast to the decay mode classification algorithm currently in use at the
ATLAS experiment, no discrimination of 1- and 3-prong modes according to the
number of tracks that are classified as \emph{charged} is made. However each
charged PFO is associated with a \emph{charged} track such that the number of
tracks is indirectly available to the network. The network is not strictly
required to use this information allowing migrations of reconstructed 1-track
(3-track) hadronic tau decays to 3-prong (1-prong) modes. If this behaviour is
not desired then the decay mode can be classified as the mode with the highest
probability that is still compatible with the number of reconstructed tracks
\todo{Compare two migration matrices. One with the track requirement one
  without}.

\begin{table}[htb]
  \centering
  \begin{tabular}{SS[table-format=2.2(2)]}%,table-space-text-post = \si{\meter}]}
  \toprule
  {$p_\text{T}$-cut / \si{\giga\electronvolt}} & {Diagonal efficiency / \si{\percent}} \\
  \midrule
  {--} & 78.41 \pm 0.06 \\
  1.0 & 78.07 \pm 0.05 \\
  1.5 & 77.95 \pm 0.04 \\
  2.0 & 77.78 \pm 0.05 \\
  2.5 & 77.48 \pm 0.05 \\
  \bottomrule
\end{tabular}

%%% Local Variables:
%%% mode: latex
%%% TeX-master: "../mythesis"
%%% End:

  \caption{Diagonal efficiency on an independent testing sample as a function of
    the transverse momentum threshold on the neutral PFOs. A separate network is
    trained for each threshold.}
  \label{tab:neut_ptcut}
\end{table}

Reconstructed hadronic tau decays contain neutral PFOs that are not originating
from neutral pions but from other sources. Moreover it is not known how well
these PFOs are modelled in the simulation that is used to train and evaluate the
models. Therefore the neutral PFOs used in the RNN are required to pass a
$p_\text{T}$-threshold. Table~\ref{tab:neut_ptcut} summarises the diagonal
efficiency of the classification for four different $p_\text{T}$-thresholds. The
diagonal efficiency is defined as the fraction of correctly classified tau
decays in an independent testing sample.
It is observed that the overall classification performance does not degrade when
no $p_\text{T}$-cut is applied, indicating that the network is insensitive to
neutral PFOs with low momentum from sources other than a $\pi^0$-decay. This is
expected as the network has access to the $\pi^0$-identification score and
transverse momentum of the PFO. Moreover, only a slow decrease in the diagonal
efficiency is observed when increasing the threshold to up to \SI{2}{\GeV}
implying that a momentum threshold on neutral PFOs in case of significant
mismodelling will not lead to a significant decrease in classification
performance. The studies in this thesis use a \SI{1.5}{\giga\electronvolt} to
account for potential mismodelling but further studies are needed to prove the
necessity of this threshold.

In Figure~\ref{fig:mode_proba_ptcut} two examples of mode probability estimates
of the RNN on an independent testing sample are depicted. A complete set of
probabilities for all five decay modes is omitted for brevity and can be found
in Appendix~\ref{app:baseline_probabilities}. The $h^\pm$ mode probability in
Figure~\ref{fig:1p0n_proba} shows that in most cases the probabilities for
generated modes containing one or more neutral pions are small. However if no
neutral PFOs are reconstructed then the probability can be as large as
\SI{90}{\percent}. In these cases the decay is falsely classified as~$h^\pm$ as
the probability is larger than \SI{50}{\percent}. The same feature can be
observed in Figure~\ref{fig:1p1n_proba}, where the $h^\pm \pi^0$ mode
probability is small. Additionally, the probabilities for generated 1-prong
decays with more than one neutral pion have large tails exceeding
\SI{50}{\percent} leading to significant migrations from $h^\pm \geq 2 \pi^0$ to
the $h^\pm \pi^0$ decay mode. \todo{Ref to next section}

\begin{figure}[ht]
  \begin{subfigure}[t]{0.48\textwidth}
    \centering
    \includegraphics{./figures/decay_mode_classification/mode_proba_baseline_ptcut_1_5_only_1p/proba_1p0n.pdf}
    \vspace*{-1.6em}
    \subcaption{}
    \label{fig:1p0n_proba}
  \end{subfigure}\hfill
  \begin{subfigure}[t]{0.48\textwidth}
    \centering
    \includegraphics{./figures/decay_mode_classification/mode_proba_baseline_ptcut_1_5_only_1p/proba_1p1n.pdf}
    \vspace*{-1.6em}
    \subcaption{}
    \label{fig:1p1n_proba}
  \end{subfigure}
  \caption{Mode probabilities for the $h^\pm$ and $h^\pm \pi^0$ modes estimated
    by the RNN for a given true decay mode. Modes with three charged hadrons
    have small $h^\pm$ and $h^\pm \pi^0$ probabilities and are omitted for
    clarity.}
  \label{fig:mode_proba_ptcut}
\end{figure}

The migration probability of generated decay modes to a given reconstructed mode
can be summarised in the so called \emph{migration matrix}. In
Figure~\ref{fig:migmat_comparison_baseline_15cut} migration matrices for both
the RNN as well as the ATLAS default algorithm is depicted. The diagonal entries
give the efficiencies to reconstruct the corresponding decay mode while the
off-diagonal measures the migration probability. The diagonal efficiency can
also be interpreted as the mean of the diagonal entries of this matrix weighted
by the reconstructed mode fractions (cf.\ Table~\ref{tab:mode_reco_eff}). The
individual efficiencies and migration probabilities in the matrix can vary
between different trainings of the network as a decrease in efficiency in one
mode can be compensated by an improvement in another. Nevertheless the diagonal
efficiency is a stable metric between different trainings and is used as a
figure of merit for subsequent investigations.

\begin{figure}[htb]
  \begin{subfigure}[t]{0.48\textwidth}
    \centering
    \includegraphics{./figures/decay_mode_classification/mig_mat_pantau.pdf}
    \subcaption{ATLAS default algorithm: \emph{PanTau}}
  \end{subfigure}\hfill
  \begin{subfigure}[t]{0.48\textwidth}
    \centering
    \includegraphics{./figures/decay_mode_classification/mig_mat_baseline_ptcut_1_5.pdf}
    \subcaption{RNN with neutral PFO $p_{\text{T}}$-cut of
      \SI{1.5}{\GeV}}
  \end{subfigure}
  \caption{Migration matrices showing the migration probabilities in
    \si{\percent} of the true decay modes to the reconstructed modes. Evaluation
    on an independent testing sample. Statistical fluctuations due to limited
    sample size can be neglected.}
  \todo[inline]{After reconstruction in
    $\gamma^* \rightarrow \tau \tau \, \text{hadr.}$ and omitting decays
    containing neutral kaons.}
  \label{fig:migmat_comparison_baseline_15cut}
\end{figure}

In comparison with the ATLAS default algorithm the RNN significantly improves
the reconstruction efficiencies of all five decay modes leading to an overall
increase in diagonal efficiency by \SI{4.8}{\percent}. Large improvements are
observed in the efficiencies for the $h^\pm$, $h^\pm \geq 2\pi^0$ and
$3h^\pm \geq 1\pi^0$ modes. Furthermore the RNN shows a significant reduction in
migrations due to under- or overestimation of the number of neutral pions.
Migrations between 1- and 3-prong modes are of similar magnitude for both the
ATLAS algorithm and the RNN classification \todo{Not true as there is a
  significant improvement in 1-prong to reco 3-prong migration, indicating that
  decay mode classification according to classified tracks is suboptimal}.

\begin{figure}[htb]
  \begin{subfigure}[t]{0.48\textwidth}
    \centering
    \includegraphics{./figures/decay_mode_classification/comp_mat_pantau.pdf}
    \subcaption{ATLAS default algorithm: \emph{PanTau}}
  \end{subfigure}\hfill
  \begin{subfigure}[t]{0.48\textwidth}
    \centering
    \includegraphics{./figures/decay_mode_classification/comp_mat_baseline_ptcut_1_5.pdf}
    \subcaption{RNN with neutral PFO $p_{\text{T}}$-cut of
      \SI{1.5}{\giga\electronvolt}}
  \end{subfigure}
  \caption{Purity matrices showing the fractional composition of the
    reconstructed decay modes in \si{percent}. Evaluation on an independent
    testing sample. Statistical fluctuations due to limited sample size can be
    neglected.}
  \label{fig:puritymat_comparison_baseline_15cut}
  \todo[inline]{Check whether the 'diagonal efficiency' is correct for
    composition matrices.}
\end{figure}

Due to the the reduction in migrations between different number of neutrals an
overall improvement in purity of the reconstructed modes is expected. This is
shown in the so called \emph{purity matrix} in
Figure~\ref{fig:puritymat_comparison_baseline_15cut} where each row gives the
fractional composition of the reconstructed decay modes and the diagonal the
corresponding mode purity. Compared to the ATLAS algorithm, the mode
classification using the RNN shows significant improvements in reconstructed
mode purity due to the decreasing migration probability between modes with
different numbers of neutral pions. The largest improvements in purity are
observed in modes containing one or more neutral pions with the largest being in
the $h^\pm \geq 2\pi^0$ mode with an absolute increase of \SI{9.7}{\percent}.


\subsection{Additional Information}
\label{sec:add_info}

This section aims to increase the classification power of the RNN by including
additional information in the network. For this the focus lies on the inclusion
of conversion tracks, shots and additional cluster moments. The following
summarises the intention behind these sources of information:
\begin{description}
\item[Conversion tracks] Often photons from the $\pi^0$ decay convert into
  electron-positron pairs in the material of the inner detector. These pairs are
  bent from the direction of the initiating photon in the magnetic field of the
  inner detector. This can weaken the signature of neutral pions in the decay as
  the energy can be split into different PFOs, affecting the
  $\pi^0$~identification score and the number of associated shots. In some cases
  the algorithm can fail to reconstruct neutral PFOs. This can lead to an
  underestimation of the number of neutral pions.

  The inclusion of conversion tracks in the classification aims to reduce
  migrations to decay modes with fewer neutrals. For this tracks classified as
  \emph{conversion} according to the track classification are used.

\item[Shots] Due to the boosted topology of tau decays in ATLAS (especially at
  large transverse momentum) neutral pion clusters can merge with other charged
  or neutral pion clusters. Without additional information two merged neutral
  pion clusters cannot be distinguished using purely kinematic quantities of the
  PFOs leading to a underestimation of the number of neutrals. To improve the
  discrimination of merged neutral pion clusters the finely segmented strip
  layer in the calorimeter is used. For this the network is extended to also use
  a sequence of shots associated with a given reconstructed tau decay~(cf.\
  Section~\ref{sec:shot_reco}). This differs from the approach in
  Section~\ref{sec:pfo_baseline}, where only the number of shots associated with
  a neutral PFO is used, as it allows including transverse momenta and angular
  distances of shots.

  This further aims to reduce migrations to modes with fewer neutrals.

\item[Neutral PFO cluster properties] Further improvements in classifying events
  with merged clusters can be achieved by employing shower shape information of
  the clusters used for PFO reconstruction. These clusters are created during
  the PFO reconstruction (cf.\ Section~\ref{sec:tau_pflow}) using cells in the
  Presampler, EM1 and EM2. Moments and other properties of the clusters, similar
  to the discriminants used for $\pi^0$-identification, are used to extend the
  input variables of neutral PFOs in the network.
\end{description}

These problems, their origin and potential solutions are discussed
in the following section.
These problems are the subject of the following section.
Challenges: No reconstructed PFOs, migrations between decay modes with one or
more than one neutrals.

\todo[inline]{These are cluster moments of the reclustered cells in PS, EM1 and
  EM2.}

\subsubsection{Conversion Tracks}
What variables, how many tracks, which tracks (400MeV reco threshold)

Up to 4 tracks (only roughly optimised by hand)

\subsubsection{Shots}
What variables, how many shots, which shot selection (Nphoton > 0,
$\eta$-dependent $E_\text{T}$ cut -- whats the intention of this cut?)

Up to 6 shots (only roughly optimised by hand)

\subsubsection{Neutral PFO cluster properties}
What variables, number of neutral pfos unchanged

\begin{description}
\item[Lateral shower width $\langle R^2 \rangle$] Second moment of the radial
  distance $R$ between cluster cells and shower
  axis~\cite{atlas_topoclustering}.

\item[$\eta$ width in EM1 $\langle (\eta - \eta_\text{cluster})^2\rangle$]
  Second moment of the pseudorapidity distance between cluster cells in EM1 and
  the energy barycentre of the
  cluster~$\eta_\text{cluster}$~\cite{atlas:taurec:decaymodes}.

\item[Number of positive cells in EM1 $N_\text{EM1}^\text{pos}$] Number of cells
  with positive energy in EM1~\cite{atlas:taurec:decaymodes}.

\item[Core energy fraction $f_\text{core}$] Fraction of the total cluster energy
  contained in the highest energetic cell in Presampler, EM1 and
  EM2~\cite{atlas_topoclustering}.

\item[Energy fraction in EM2 $f_\text{EM2}$] $f_\text{EM2} = E_\text{EM2} / E$
  \todo{Highly anti-correlated with EM1 therefore only one is picked.}

\item[Fraction of subtracted transverse momentum $f_\text{sub}$] Fraction of
  total cluster energy subtracted during reconstruction of the particle flow
  object
  \begin{align*}
    f_\text{sub} = \frac{p_\text{T}^\text{cluster} - p_\text{T}^\text{PFO}}{p_\text{T}^\text{cluster}} \eqdot
  \end{align*}

\end{description}
Longitudinal moments were not found to lead to an increase performance.

\begin{table}[htb]
  \centering
  \begin{tabular}{p{5cm}S[table-format=1.4(4)]S[table-format=2.2(2)]S[table-format=1.2(2)]}
  \toprule
  {Additional input} & {Loss} & {Diag.\ eff.\ / \si{\percent}} & {Diag.\ eff.\ gain (abs.) / \si{\percent}} \\
  \midrule
  Conversion tracks & 0.5186 +- 0.0013 & 79.40 +- 0.07 & 1.45 +- 0.08 \\
  Conversion tracks (extrapol.) & 0.5224 +- 0.0012 & 79.23 +- 0.06 & 1.28 +- 0.07 \\
  Shots & 0.5239 +- 0.0011 & 79.52 +- 0.06 &  1.57 +- 0.07 \\
  Neut.\ PFO cluster properties & 0.5310 +- 0.0010 & 79.07 +- 0.06 & 1.12 +- 0.07 \\
  Hadronic PFOs & 0.5433 +- 0.0007 & 78.30 +- 0.04 & 0.35 +- 0.06 \\
  Subtracted $p_\text{T}$-fraction & 0.5466 +- 0.0005 & 78.26 +- 0.03 & 0.31 +- 0.05 \\
  $\pi^0$-BDT ordering & 0.5511 +- 0.0013 & 78.01 +- 0.07 & 0.06 +-0.08 \\
  \bottomrule
\end{tabular}

%%% Local Variables:
%%% mode: latex
%%% TeX-master: "../mythesis"
%%% End:

  \caption{Additional experiments}
  \label{tab:pfo_add_experiments}
\end{table}

\todo[inline]{Try out bidirectional LSTMs and stacked ones (still not
  overfitting).Technically would allow multiple outputs: Could do Pi0 Cluster-ID
  internally. What else has been tested?}

\subsection{Extended Model}

\begin{figure}[!ht]
  \begin{subfigure}{0.48\textwidth}
    \centering
    \includegraphics{./figures/decay_mode_classification/combined_sub_e_moments_shots_conv_ptcut_1_5/mig_mat.pdf}
    \subcaption{Migration matrix}
  \end{subfigure}\hfill
  \begin{subfigure}{0.48\textwidth}
    \centering
    \includegraphics{./figures/decay_mode_classification/combined_sub_e_moments_shots_conv_ptcut_1_5/comp_mat.pdf}
    \subcaption{Purity matrix}
  \end{subfigure}
  \caption{Combined}
  \label{fig:decay_mode_combined}
\end{figure}

\begin{figure}[!ht]
  \begin{subfigure}{0.48\textwidth}
    \centering
    \includegraphics{./figures/decay_mode_classification/combined_sub_e_moments_shots_conv_ptcut_1_5/efficiency_profile.pdf}
    \subcaption{Efficiency profile}
  \end{subfigure}\hfill
  \begin{subfigure}{0.48\textwidth}
    \centering
    \includegraphics{./figures/decay_mode_classification/combined_sub_e_moments_shots_conv_ptcut_1_5/purity_profile.pdf}
    \subcaption{Purity profile}
  \end{subfigure}
  \caption{$p_\text{T}$-dependence of mode efficiency and purity}
  \label{fig:mode_efficiency_purity}
\end{figure}

\todo[inline]{Class probabilities (showing the non-existent peak). Improvements
  for Higgs CP measurement in
  $H \rightarrow \tau\tau \rightarrow \rho \rho 2\nu$. Performance when applying
  medium tau identification. Fluid classification -- a single PFO is not
  'causing' a particular classification result -- could use heuristic methods to
  reconstruct neutral pions after classification. Unleash on data without upper
  $p_\text{T}$ limit.}


\begin{figure}[!ht]
  \begin{subfigure}{0.48\textwidth}
    \centering
    \includegraphics{./figures/decay_mode_classification/combined_sub_e_moments_shots_conv_ptcut_1_5/mig_mat_med_id.pdf}
    \subcaption{Migration matrix}
  \end{subfigure}\hfill
  \begin{subfigure}{0.48\textwidth}
    \centering
    \includegraphics{./figures/decay_mode_classification/combined_sub_e_moments_shots_conv_ptcut_1_5/comp_mat_med_id.pdf}
    \subcaption{ Purity matrix}
  \end{subfigure}
  \caption{Combined with medium tau id}
  \label{fig:decay_mode_combined_med_id}
\end{figure}

\clearpage
\subsection{High-$p_\text{T}$ performance}
\begin{figure}[htb]
  \begin{subfigure}{0.48\textwidth}
    \centering
    \includegraphics{./figures/decay_mode_classification/highpt/efficiency_profile.pdf}
    \subcaption{Efficiency profile}
  \end{subfigure}\hfill
  \begin{subfigure}{0.48\textwidth}
    \centering
    \includegraphics{./figures/decay_mode_classification/highpt/purity_profile.pdf}
    \subcaption{Purity profile}
  \end{subfigure}
  \caption{$p_\text{T}$-dependence of mode efficiency and purity}
  \label{fig:mode_efficiency_purity_highpt}
\end{figure}

\begin{figure}[htb]
  \begin{subfigure}{0.48\textwidth}
    \centering
    \includegraphics{./figures/decay_mode_classification/highpt/mig_mat_pt_geq_100.pdf}
    \subcaption{Migration matrix. $p_\text{T} > \SI{100}{\giga\electronvolt}$}
  \end{subfigure}\hfill
  \begin{subfigure}{0.48\textwidth}
    \centering
    \includegraphics{./figures/decay_mode_classification/highpt/comp_mat_geq_100.pdf}
    \subcaption{Purity matrix. $p_\text{T} > \SI{100}{\giga\electronvolt}$}
  \end{subfigure}
  \caption{highpt}
  \label{fig:highpt_matrices}
\end{figure}


%%% Local Variables:
%%% mode: latex
%%% TeX-master: "mythesis"
%%% End:
