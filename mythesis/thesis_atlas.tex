% ~ 6 pages
\chapter{The ATLAS Experiment at the Large Hadron Collider}
\label{chap:atlas}

\section{The Large Hadron Collider}
\label{sec:lhc}

\section{The ATLAS Experiment}
\label{sec:atlas}

\subsection{Coordinate System}
\label{sec:atlas_coord_sys}

\begin{itemize}
\item $\eta$
\item $\Delta R$
\item $p_\mathrm{T}$
\end{itemize}

\subsection{Tracking System}
\label{sec:atlas_tracking}

Name this 'Inner Detector'?

\todo{Find primary sources on this stuff!}
\begin{itemize}
\item Track \& vertex reconstruction
\item Charged particles provide hits in ID
\item High spacial resolution to reconstruct tracks in dense environments
  (number of tracks per event -- 200? -- check this)
\item Solenoid field with \SI{2}{\tesla}
\item Secondary vertex reconstruction (B-layer -- this is important, Pixel, SCT)
\item TRT -- 'continuous' tracking (what is this?) and distinction between
  electrons and hadrons using transition-radiation
\item ID acceptance $|\eta| < \num{2.5}$
\item Split into barrel and two endcap regions
\item Pixel: Three barrel layers, three endcap layers each (and 1 layer IBL).
  Pixel detector in 'Pixel Support Tube' to allow exchange of elements in case
  of radiation damage. Inner Radius \SI{4.2}{\centi\metre} and outer radius
  \SI{25}{\centi\metre}.
\item The layer after the IBL is called B-layer!
\item IBL \cite{ibl_tdr}: Improve B-tagging when modules in the remaining pixel layers fail;
  Tracking inefficiencies at high pileup affects B-layer needs redundancy;
  Better impact parameter reconstruction improves vertexing and b-tagging
\item Pixel layers segmented in $r\varphi$ and $z$. Single pixel
  \SI{50}{\micro\metre} in $r\varphi$ and \SI{400}{\micro\metre} in $z$
\item Silicon strip detectors (four layers) in support structure with radius of
  about \SI{55}{\centi\metre}
\item SCT strip detectors \SI{15}{\micro\metre} $r\varphi$ and
  \SI{70}{\micro\metre} $z$ resolution
\item TRT: Barrel region $|\eta| < \num{0.5}$, on average a track causes 36
  hits; Resolution decreases with pile-up
\end{itemize}

\subsection{Calorimeter System}
\label{sec:atlas_calo}



\begin{itemize}
\item LHC (brief)

\item ATLAS
  \begin{itemize}
  \item Overview (Design goals) \\
    brief: Beam Line, Inner Detector \& Solenoid, Calorimeter, Muon System \&
    Toroid, Trigger

  \item Nomenclature (Coordinate system, Pseudorapidity, $\Delta R$,
    $p_\mathrm{T}$)

  \item Inner detector / Tracker (and why they are important for taus)
    \begin{itemize}
    \item Pixel, IBL
    \item SCT
    \item TRT
    \item  Transverse Momentum Resolution, Vertex \& Secondary Vertex
      reconstruction, Impact Parameter Resolution, $\eta$-Coverage
    \end{itemize}

  \item Calorimeter (and why they are important for taus)
    \begin{itemize}
    \item Presampler, LAr (EM1 - EM3), Had (Tile, LAr)
    \item Cell sizes, $\eta$-Coverage, Thickness $X_0$ / $\lambda$,
      Energy Resolution vs. $E$
    \item Topoclusters \& Cluster moments
    \end{itemize}

  \end{itemize}
\end{itemize}

%%% Local Variables:
%%% mode: latex
%%% TeX-master: "mythesis"
%%% End:
