% ~ 6 pages
\chapter{Theoretical Background}
\label{sec:theory}

In this chapter the theoretical background for the studies in this thesis is
provided. First the Standard Model of particle physics is introduced and its
problems discussed. Then physics of tau leptons is discussed with a focus on
hadronic decays of the tau lepton.

\section{The Standard Model of particle physics}

The Standard Model of particle physics describes the current understanding of
elementary particles and interactions between them. The Standard Model has been
extensively probed since its origin and shows excellent agreement with
experimental results. However, it does not offer a full description of particle
physics as it does not explain interactions of massive particles via gravity.

In Figure~\ref{fig:sm_particles} the elementary particles of the SM are shown.
They can be divided into the fermions consisting of particles with spin-half and
gauge with spin-1. In 2012 the ATLAS and CMS experiment at the Large Hadron
Collider discovered a new scalar boson with a mass
of~$m_\text{H} \approx \SI{125}{\GeV}$~\cite{higgs_atlas, higgs_cms}, which is
called the Higgs boson.

In the following the elementary particles and their interactions are described.
The fermions of the Standard Model form the observable matter. The dynamics of
fermions are described by the Dirac equation, while interactions between them
are mediated by the gauge bosons. The nature of these interactions is described
by quantum field theories and the local gauge principle. The fermions are
divided into up- and down-type quarks with a charge of 2/3 and -1/3,
respectively and leptons consisting of the charged leptons and the neutrinos.
The fermions are divided into three generations with identical properties with
the exception of their mass. The first generation consisting of the electron,
electron neutrino, up- and down-quark makes up most of the matter in the
low-energy universe. Each fermion has a corresponding antiparticle with the same
mass but with inverted additive quantum numbers including the electric charge.

The strong interaction between quarks is described by quantum chromodynamics.
The exchange particle of the strong force is the massless gluon, which couples
to the so called color charge. The quarks as well as the gluons carry color
charge allowing $qqg$-vertices as well as gluon self-coupling in triple and
quartic gluon vertices. Experimentally, free quarks are not observed but only
color-neutral bound states called hadrons. This is known as colour confinement
and leads to the formation of collimated jets of hadrons in processes creating
high-energy quarks or gluons.

The electromagnetic interaction between particles carrying electric charge is
described by quantum electrodynamics. The mediating gauge boson is the massless
photon, which couples to charged particles including quarks, charged leptons and
the $W^\pm$~boson.

The weak interaction

parity violating

charged-current interaction
neutral-current interaction





Weak interaction of all fermions - W, Z bosons

Unification of weak and electromagnetic force into the electroweak force





higgs boson quantum of the higgs field generating masses of elementary particles

Higgs mechanism
Higgs mechanism / electroweak symmetry breaking: generate particle masses\\

\begin{figure}[htb]
  \centering
  \includegraphics[width=0.7\textwidth]{./figures/theory/sm_particles.pdf}
  \caption{Particles of the Standard Model. Image modified from
    Ref.~\cite{sm_wiki}.}
  \label{fig:sm_particles}
\end{figure}

Parameters of the Standard Model: 12 fermion masses (Higgs sector), 3 coupling
constants (gauge interactions) $\alpha$, $G_{\text{F}}$, $\alpha_{\text{S}}$,
VEV~$v$ and $m_{\text{H}}$ (Higgs sector), 8 mixing angles of the PMNS and CKM
matrices (flavour sector), CP violating phase of the strong interaction
\cite{thomson}.
Large number of parameters 25 (26 counting the CP violating phase) with 14
related to the Higgs field.

\subsection{Problems of the Standard Model}

Challenges: Dark matter, Gravity

Most common high-energy process at the LHC: Production of dijet-events (back to
back in the transverse plane). A lot of QCD diagrams contribute to the dijet
cross-section (qq \textrightarrow qq, qq \textrightarrow gg, gg \textrightarrow
qq, qg \textrightarrow gq, gg \textrightarrow gg, etc.). Dijet cross-section
peaked at low $p_{\text{T}}$.


\section{Tau Leptons}

\todo[inline]{Define 1-prong, 3-prong and multi-prong.}

The tau lepton is the heaviest lepton in the Standard Model and an important
probe of physics at high energy scales, such as Higgs physics and physics beyond
the Standard Model. Hadronic decays make up approximately two-thirds of the
total branching ratio of tau decays and play an important part in the physics
programme of the ATLAS experiment.

\begin{figure}[ht]
  \begin{subfigure}[b]{0.47\textwidth}
    \centering
    \begin{overpic}{./figures/theory/tau_branching_pie_chart.pdf}
      \put (34, 83) {$\pi^- \nu_\tau$}
      \put (-1, 45) {$\pi^- \pi^0 \nu_\tau$}
      \put (19, 8) {$\pi^- 2 \pi^0 \nu_\tau$}
      \put (44.5, 2) {$2 \pi^- \pi^+ \nu_\tau$}
      \put (69, 6) {$2 \pi^- \pi^+ \pi^0 \nu_\tau$}
      \put (80, 15) {other}
    \end{overpic}
    \caption{Tau branching ratios \cite{pdg}. Adapted from
      Ref.~\cite{ikai_trigger}.}
    \todo[inline]{Fix wonky percentages for hadronic modes}
    \label{fig:tau_branching_ratios}
  \end{subfigure}\hfill
  \begin{subfigure}[b]{0.47\textwidth}
    \centering
    \includegraphics{./figures/theory/tau_decay_feynman.pdf}
    \caption{Feynman Diagram. Strangeness production is Cabibbo suppressed by a
      factor of $\sin^2\theta_\mathrm{c} / \cos^2\theta_\mathrm{c} \approx
      \frac{1}{20}$. Ignoring phase space considerations the tau decays to
      $\frac{1}{5}$ into electron or muon and corresponding antineutrino or into
      a down anti-up quark pair (3 colours).}
  \end{subfigure}
  \caption{Weak decay of the tau-lepton}
\end{figure}

\begin{itemize}
\item Discovery at SPEAR in 1975 [Check Citation]\cite{perl}
\item $m_\tau = \SI{1776.86 +- 0.12}{\mega\electronvolt}$ \cite{pdg}
\item Tau more than twelve times heavier than pions \textrightarrow can decay
  into quark-antiquark pairs (strangeness production cabbibo suppressed)
\item 5 possible weak decay modes: electron (20\%), muon (20\%), ud (3 times
  20\% due to 3 color charges) -- ignoring phase space
\item unit charge \textrightarrow decays into odd number of charged particles
  (prong definition)
\item Two or three pion modes mainly through intermediate $\rho$ or $a_1$
  resonances
\item Charged pion 'stable' and showers in the HCAL
\item Neutral pion decays immediately into two photons which shower in the ECAL.
  Created photons often convert into an electron positron pair in the presence
  of material
\end{itemize}

\todo[inline]{Introduce \tauhad and \tauhadvis notation}

\section{Features of Hadronically Decaying Tau Leptons}
\label{sec:features_tau_decay}

\todo[inline]{Ignore non 1- and 3-prong modes.}

\todo[inline]{Signature of the decay modes}

\todo[inline]{Leptonic decays of the tau are identified using the electron/muon
  identification.}

\todo[inline]{Quark-like jets are on average more collimated and have fewer
  tracks and thus the discrimination from \tauhadvis is less effective than for
  gluon-like jets. Due to higher effective colour charge for gluons.}

\todo[inline]{Define what 'visible' means}

Features of hadronically decaying $\tau$-leptons vs. Quark/Gluon initiated jets:
\begin{itemize}
\item Low multiplicity (QCD jets usually have a lot of tracks)
\item Isolated tracks and narrow showers
\item Measurable decay length with proper lifetime of
  $\tau = \SI{290.3 +- 0.5}{\femto\second}$ \cite{pdg} and following from that
  $c \tau = \SI{87.0 +- 0.2}{\micro\metre}$ with $\beta \gamma = 10$ which
  corresponds to a roughly \SI{18}{\giga\electronvolt} tau the mean decay length
  is of the order of a millimetre and allows secondary vertex reconstruction or
  employing the impact parameter for 1-prong taus (sub-millimetre resolution).
\item Invariant mass bounds for decay products
\item $\pi^0$ content???
\end{itemize}

% ----------------------- FIRST DRAFT ---------------------- %
\begin{itemize}

\item The Standard Model
  \begin{itemize}
  \item Features \& Successes

  \item Challenges (neutrino masses, dark matter, matter-antimatter asymmetry,
    gravitation, number of parameters, hierarchy problem, \ldots)

  \item Beyond the Standard Model (SUSY -- preferred coupling to down-type
    fermions for large $\tan\beta$ \textrightarrow $\tau$-leptons)
  \end{itemize}

\item Weak Interaction
\begin{itemize}
\item Can be kept very short. Only whats necessary to understand Tau-ID
\end{itemize}

\item Strong Interaction
\begin{itemize}
\item Can be kept very short. Only whats necessary to understand Tau-ID
\item Confinement \& Hadronization
\item Quark \& Gluon initiated jets
\end{itemize}

\item $\tau$-Leptons
\begin{itemize}
\item Discovery

\item Properties (mass \textrightarrow lep \& had, mean life time
  \textrightarrow no direct detection)

\item Hadronically Decaying $\tau$-Leptons
  \begin{itemize}
  \item Feynman Diagram, Decay Modes (interm. resonances) \& Branching Ratios
    (pie chart like in: \url{https://www.lhc-ilc.physik.uni-bonn.de/ergebnisse/dateien/t00000078.pdf?c=t&id=78}
    on p.\ 25)
  \item Detector signature ($\pi^0$ ($\gamma \gamma$ / $\mathrm{e}^+
    \mathrm{e}^- \gamma$), $\pi^\pm$ ($\mathrm{K}^\pm$), $\nu_\tau$,
    conversions)
  \item Jets faking taus
  \end{itemize}

\item $\tau$ Physics
  \begin{itemize}
  \item $\mathrm{Z} \rightarrow \tau \tau$ (background for H$\tau \tau$ and
    useful for performance measurements using tag-and-probe -- semileptonic
    decays)

  \item $\mathrm{H} \rightarrow \tau \tau$ (one of two channels to measure
    the fermionic coupling -- $b \bar{b}$ plagued by multijet background,
    Higgs CP)

  \item MSSM Higgs (potentially high branching fraction to $\tau$-leptons)

  \item $\mathrm{Z}^\prime$ could preferentially decay into third-generation
    fermions (lepton universality not required).

  \item $\mathrm{W}^\prime$ models with preferential coupling to third-gen.

  \item SUSY with $\tau$ \textrightarrow long decay chains
  \end{itemize}
\end{itemize}
\end{itemize}

% -----------------------------???-------------------------- %

\begin{itemize}
\item Weak \& strong interaction (?)
  \begin{itemize}
  \item Cross section plot multijet (large cross section) vs.
    electroweak interaction (small)
  \item \url{http://www.hep.ph.ic.ac.uk/~wstirlin/plots/plots.html}
  \item Quark vs Gluon jets: \url{http://jets.physics.harvard.edu/qvg/}
    Because of different colour interaction and hadronization, gluon jets are
    wider, with higher multiplicity and have a more uniform energy
    fragmentation, while quark jets are more likely to produce narrow jets with
    hard constituents that carry a significant fraction of the energy.
  \end{itemize}

\item Tau decay modes \& branching fractions
  \begin{itemize}
  \item Hadronic decays not containing pions but kaons etc.
  \end{itemize}

\item Intermediate resonances $a$, $\rho$

\end{itemize}

%%% Local Variables:
%%% mode: latex
%%% TeX-master: "mythesis"
%%% End:
